\documentclass{article}
\usepackage[utf8]{inputenc}
\usepackage{graphicx}
\usepackage{amsmath}
\usepackage[margin=1.0in]{geometry}


\title{316 Notes: Lecture 1}
\author{Ashtan Mistal}
\date{May 10 2021}

\begin{document}

\maketitle


Some basic, quick notes regarding the course:

\begin{itemize}
    \item Watch the pre-recorded lectures prior to the actual lectures
    \item Homeworks: 20\%; Midterm: 30\%; Final: 50\%
    \item Homeworks are best 4 out of 5
    \item Exams will be open book, open notes
    \item Homeworks can be written + scanned, or written using OneNote or \LaTeX
\end{itemize}

\section{Brief Overview of ODEs}

First Order ODEs

\begin{itemize}
    \item Separable ODEs (ex: $y' = \frac{dy}{dx} = p(x) \cdot q(y)$
    \item Linear: $Ly = y' + p(x) y = Q(x)$, where $\underbrace{L}_{\text{linear}} = \frac{d}{dx} + p(x)$
\end{itemize}

Second Order ODEs:

\begin{itemize}
    \item Constant coefficient ODE: $Ly = y'' + a y' + by = 0$
    \item Cauchy-Euler, or Equi-dimensional ODEs: $Ly = x^2 y'' + a x y' + by = 0$
    \item L is the linear operator. In the constant coefficient, $L = \frac{d^2}{dx^2} + a \frac{d}{dx} + b$
\end{itemize}

\subsection{Examples}
\subsubsection{First Order Separable Equation Examples}

$\frac{dy}{dx} = p(x) Q(y) \longrightarrow \int{\frac{dy}{Q(y)}} = \int{P(x) dx + C}$

Then, we get:

$\frac{dy}{dx} = y \cos(x) \rightarrow \frac{dy}{y} = \cos(x) dx \longrightarrow \ln{|y|} = \sin(x) + C \rightarrow y = C_1 e^{\sin(x)}$

\subsubsection{Linear First Order Equation:}

$y' + p(x) y = Q(x)$

Multiply both sides by an integrating factor $\mu(x)$

Ex: $y' + \frac{2x}{1 + x^2} y = \frac{\cot(x)}{1 + x^2}$
$\mu(x) y' + \underbrace{\mu(x) \frac{2x}{1 + x^2} y}_{\mu'(x) y} = \frac{\cot(x)}{1 + x^2} \mu(x)$
Compare to: $\mu(x) y' + \mu'(x) y = \left[\mu(x) y(x) \right]'$

$\frac{d \mu}{dx} = \mu \frac{2x}{1 + x^2}$: integrate: $\int{\frac{d \mu}{\mu}} = \int{\frac{2x}{1 + x^2} dx + C}$

Hence, using integrating factor: $$ \ln |\mu(x)| = \ln(1 + x^2) + C$$ with $\mu(x) = C_1 (1 + x^2)$

Hence, $$C_1(1 + x^2) y' + C_1 2xy = C_1 \cot(x)$$

$$\frac{d}{dx} \left[ (1 + x^2) y \right] = \cot(x)$$

$$(1+x^2)y = \int \cot(x) + C = \ln|\sin(x)| + C$$

Hence, $$y(x) = \frac{\ln|\sin(x)|}{1+x^2} + \frac{C}{1+x^2}$$

\subsection{Second Order Constant Coefficient ODE}

$$ay'' + by' + cy = 0$$

We start with a guess: $e^{rx}$, and substitute: $(ar^2 + br + c) \cdot e^{rx} = 0$. Note that $ar^2 + br + c$ is the characteristic equation. We then have two solutions:'

$$r_{1,2} = \frac{-b \pm \sqrt{b^2 - 4ac}}{2a}, \Delta = b^2 - 4ac$$

Based on the sign of $Delta$, we have 3 different cases:

\begin{itemize}
    \item $\Delta > 0$: 2 real, distinct roots
    \item $\Delta < 0$: 2 complex roots
    \item $\Delta = 0$: Repeated roots.
\end{itemize}

Example: $2y'' + 2y' + y = 0$: Guess $e^{rx}$.

$2r^2 + 2r + 1 = 0 \longrightarrow r_{1,2} = \frac{-2 \pm \sqrt{4 - 8}}{2(2)}, \Delta = -4$. As $-4 < 0$, this is 2 complex roots. We end up with the following solution: $$e^{\frac{-x}{2}} \left[C_1 e^{\frac{i}{2} x} + C_2 e^{\frac{-i}{2} x} \right]$$

Using Euler's formula $e^{i \theta} = \cos(\theta) + i \sin(\theta)$, we get the following: $$e^{\frac{-x}{2}} \left[ C_1 (\cos(\frac{x}{2}) + i \sin(\frac{x}{2})) + C_2 (\cos(\frac{x}{2}) - i \sin(\frac{x}{2})) \right]$$

$$e^{\frac{-x}{2}} \left[(C_1 + C_2) \cos(\frac{x}{2}) + i (C_1 + C_2) \sin(\frac{x}{2}) \right]$$

$$y(x) = e^{\frac{-x}{2}} \left[ A_1 \cos{\frac{x}{2}} + A_2 \sin{\frac{x}{2}} \right]$$

Real form of the solution, A1 \& A2: $c_1 = a_1 + i b_1$; $c_2 = a_2 + i b_2$

$A_1 = (a_1 + a_2) + i (b_1 + b_2)$, and $A_2 = i(a_1 - a_2) - (b_1 - b_2)$. $b_1 + b_2 = 0$, $a_1 - a_2 = 0$

\hfill \break
Example: $y'' - 2y' + y = 0$. Characteristic equation: $r^2 - 2r + 1 = 0 \longrightarrow (r - 1)^2 = 0$, and therefore the roots are $r = 1$ repeated. Hence,

$$y = C_1 e^{x} + C_2 x e^x$$

is the solution of the equation.

\subsubsection{Cauchy - Euler Eqn}

$$Ly = x^2 y'' + \alpha x y' + \beta y = 0$$

Guess: $y = x^r$

EX: $2x^2 y'' - x y' + y = 0$. $y(x) = x^r, y'(x) = r x^{r-1}, y'' = r(r-1) x^{r-2}$.

Therefore, $$2 r (r-1)x^r - rx^r + x^r = 0$$

which is equivalent to $$\left[2r(r-1) - r + 1 \right] x^2 = 0$$

$$2r(r-1) - r + 1 = 0 \longrightarrow 2r^2 - 3r + 1 = 0$$

$$r_{1,2} = \frac{3 \pm \sqrt{9 - 8}}{2(2)} = 1, \frac{1}{2}$$

Hence, the solution to the equation is $y(x) = C_1 x + C_2 x^{\frac{1}{2}}$

\hfill

EX2: $x^2 y'' - x y' + y = 0$

$y(x) = x^r$ hence, we get: $r(r-1)x^r - rx^r + x^r = 0$.

Therefore $\left[r(r-1) - r + 1 \right]x^r = 0$.

$$r^2 - 2r + 1 = (r-1)^2 = 0 \rightarrow r = 1$$

$$y_1 = x, y_2 = \ln(x) \cdot x$$

$$y(x) = C_1 x + C_2 x \ln(x)$$

(?) $L y = 0, L \frac{d}{dr} y(x,r) = 0, \frac{d}{dr}(x^r) = x^r \ln(x)$

\hfill

EX3: $x^2 y'' - x y' + 5y = 0$.
$y(x) = x^r$, and therefore $r(r-1)x^2 - r x^r + 5 x^r = 0$

$\left[r(r-1) - r + 5 \right] x^r = 0$, hence, $r^2 - 2r + 5 = 0$. We then get the general solution of the following:

$$y(x) = C_1 x^{1 + 2i} + C_2 x^{1 - 2i}$$

This can be re-written as the following:

$$y(x) = x \left[ C_1 e^{2i \ln(x)} + C_2 e^{-2i \ln(x)} \right]$$

And hence as the following:

$$y(x) = x \left[ C_1 (\cos(2 \ln (x) + i \sin(2 \ln(x)) + C_2 (\cos(2 \ln(x)) - i \sin(2 \ln(x))) \right]$$

$$ = x \left[(c_1 + c_2) \cos(2 \ln(x)) + i (c_1 - c_2) \sin(2 \ln(x)) \right]$$

$$ = x \left[A_1 \cos(2 \ln(x)) + A_2 \sin(2 \ln(x)) \right]$$

$A_1$ and $A_2$ are real.

\hfill

\hfill

EX4: Solve the IVP

$$x^2 y'' - 3 xy' + 4y = 0, y(1) = 1, y'(1) = 1$$

If we let $y(x) = x^r$, then we get the following: $y(x) = x^r, y'(x) = r x^{r-1}, y'' = r(r-1) x^{r-2}$

Plugging this into our equations:

$x^2 r(r-1) x^{r-2} - 3 xr x^{r-1} + 4 x^r = 0$

Solving, we discover we have a repeated root of 2. Hence, the general solution is $y(x) = C_1 x^2 + C_2 x^2 \ln(x)$. Plugging in the initial conditions, we find that $C_1 = 1$, and $C_2 = -1$ and hence the solution is $y(x) = x^2 - x^2 \ln(x)$

\section{Series Solutions of ODEs}

We use power series expansion to solve variable coefficient linear ODEs. Remember that a function $f(x)$ can be approximated by a polynomial of degree $n$, such that $f(x) = a_0 + a_1 x + a_2 x^2 + ... + a_n x^n$. As the degree of $n$ increases, the approximation improves. Hence, $f(x) = \sum_{m = 0}^{\infty} a_m x^m$. Or, in general, we can approximate $f(x)$ by a power series expanded about a point $x_0$, writing it as $f(x) = \sum_{n = 0}^{\infty} a_n (x - x_0)^n$. We can remember that when we had the Taylor series, $a_n = \frac{f^{(n)} (x_0)}{n!}$. So, we have: $f(x) = \sum_{n = 0}^{\infty} \frac{f^{(n)} x(0)}{n!} (x - x_0)^n$.

\subsubsection{Example: $y' + (1 - 2x) y = 0$}

Using the integrating factor, $\mu(x) = e^{\int (1 - 2x) dx} = e^{x - x^2}$.

$\left[ y e^{x - x^2} \right]' = 0 \longrightarrow y e^{x - x^2} = C \rightarrow y = C e^{-x + x^2}$.

Use Taylor expansion $y(x)$ about the point $x_0 = 0$.

In order to do this, we write it as the sum described above:

$$f(x) = \sum_{n = 0}^{\infty} \frac{f^{(n)} x(0)}{n!} (x - x_0)^n$$

We can then let $y = \sum_{n = 0}^{\infty} \frac{f^{(n)} x(0)}{n!} x^n$ and $y' = \sum_{n = 1}^{\infty} \frac{f^{(n)} x(0)}{n!} n x^{n-1}$, and therefore we can write our ODE as $$\sum_{n = 1}^{\infty} \frac{f^{(n)} x(0)}{n!} x^{n-1} + (1 - 2x)\sum_{n = 0}^{\infty} \frac{f^{(n)} x(0)}{n!} x^n = 0$$
\end{document}
